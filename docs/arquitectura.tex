\documentclass[a4paper,12pt]{article}
\usepackage[utf8]{inputenc}
\usepackage[spanish]{babel}
\usepackage{graphicx}
\usepackage{hyperref}
\usepackage{listings}
\usepackage{xcolor}
\usepackage{array}


\hypersetup{
  colorlinks=true,
  linkcolor=blue,
  citecolor=blue
}

\lstset{
  language=Java,
  basicstyle=\ttfamily\small,
  keywordstyle=\color{javapurple}\bfseries,
  stringstyle=\color{javagreen},
  commentstyle=\color{javacomment},
  numbers=left,
  frame=single,
  breaklines=true
}

\renewcommand{\lstlistingname}{Código}

\title{Simulación de Cafetería con Programación Concurrente}
\author{Pedro José Meixús Belsol}
\date{\today}

\begin{document}

\maketitle
\newpage

\tableofcontents

\newpage

\section{Introducción}

La aplicación desarrollada simula el funcionamiento de una cafetería donde los clientes llegan, esperan ser atendidos por camareros, y se marchan tras recibir su pedido o si se cansan de esperar. Esta simulación utiliza hilos en Java para representar los diferentes actores (clientes y camareros) que actúan de forma independiente. Se le ha añadido una interfaz gráfica que ayuda a su visualización.

\section{Arquitectura del Sistema}

\subsection{Diagrama de Clases}

A continuación veremos las clases principales:

\begin{itemize}
    \item \textbf{Cliente}: Representa a una persona que entra a la cafetería, espera ser atendida, y puede marcharse si espera demasiado tiempo.
    \item \textbf{Camarero}: Representa a un empleado que atiende a los clientes preparándoles café.
    \item \textbf{Cafeteria}: Gestiona la lista de clientes y la asignación de estos a los camareros.
    \item \textbf{Main}: Clase que carga el FXML e inicializa el programa.
    \item \textbf{CafeteriaTrigger}: Interfaz que manda una señal a la UI para que realice un cambio.
    \item \textbf{HelloController}: Clase que implementa la interfaz 'CafeteriaTrigger', gestiona el flujo del programa e indica que hacer cuando recibe cada trigger.
\end{itemize}

\subsection{Relaciones entre Clases}

\begin{itemize}
    \item \textbf{Cafetería} mantiene listas de clientes y gestiona su estado.
    \item \textbf{Cliente} y \textbf{Camarero} son subclases de \texttt{Thread}, lo que les permite ejecutarse como hilos independientes.
    \item Los \textbf{Clientes} se registran en la \textbf{Cafetería} al llegar.
    \item Los \textbf{Camareros} consultan a la \textbf{Cafetería} para obtener el siguiente cliente a atender.
    \item \textbf{CafeteriaTrigger} envía los triggers a \textbf{HelloController}.
    \item \textbf{HelloController} recibe los triggers de \textbf{CafeteriaTrigger} y da una orden en consecuencia.
\end{itemize}

\section{Implementación de la Concurrencia}

\subsection{Modelo de Hilos}

Cada cliente y camarero se ejecuta como un hilo independiente:

\begin{itemize}
    \item \textbf{Hilos de Cliente}: Simulan clientes que llegan a la cafetería con tiempos aleatorios, esperan ser atendidos durante un tiempo máximo predefinido, y se marchan si no son atendidos.
    \item \textbf{Hilos de Camarero}: Buscan clientes para atender, simulan la preparación del café, y entregan la bebida al cliente.
\end{itemize}

\subsection{Control de Concurrencia}

Para garantizar que dos camareros no atiendan al mismo cliente, se implementó:

\begin{itemize}
    \item \textbf{Estado de proceso}: Cada cliente tiene un atributo \texttt{enProceso} que se activa cuando un camarero comienza a atenderlo.
    \item \textbf{Lista de clientes en atención}: La cafetería mantiene un registro de los clientes que están siendo atendidos.
    \item \textbf{Secuenciación de inicio}: Los camareros se inician con una pequeña diferencia de tiempo para reducir colisiones iniciales.
\end{itemize}

\section{Interfaz Gráfica de Usuario}

\subsection{Componentes de la Interfaz}

La interfaz gráfica está implementada utilizando JavaFX y está compuesta por los siguientes elementos:

\begin{itemize}
    \item \textbf{Listas visuales (ListView)}: Seis listas que muestran el estado de los diferentes actores:
    \begin{itemize}
        \item \texttt{listaClientesLlegan}: Muestra los clientes que acaban de llegar a la cafetería.
        \item \texttt{listaClientesAtendidos}: Muestra los clientes que ya han sido atendidos completamente.
        \item \texttt{listaClientesSiendoAtendidos}: Muestra los clientes que están siendo atendidos en este momento.
        \item \texttt{listaClientesSeVan}: Muestra los clientes que se han marchado sin ser atendidos.
        \item \texttt{listaCamarerosTrabajando}: Muestra los camareros que están trabajando actualmente.
        \item \texttt{listaCamarerosTerminaron}: Muestra los camareros que han terminado su turno.
    \end{itemize}
    \item \textbf{Botón de inicio}: Permite iniciar la simulación de la cafetería.
\end{itemize}

\subsection{Flujo de Actualización de la Interfaz}

El flujo de actualización de la interfaz se gestiona mediante el patrón Observer, donde:

\begin{enumerate}
    \item La clase \texttt{CafeteriaTrigger} actúa como una interfaz que define los eventos que pueden ocurrir en la simulación:
    \begin{itemize}
        \item \texttt{clienteLlega}: Cuando un cliente nuevo llega a la cafetería.
        \item \texttt{clienteSiendoAtendido}: Cuando un camarero comienza a atender a un cliente.
        \item \texttt{clienteSeVa}: Cuando un cliente abandona la cafetería sin ser atendido.
        \item \texttt{clienteAtendido}: Cuando un cliente ha sido atendido completamente.
        \item \texttt{camareroTrabajando}: Cuando un camarero está trabajando.
        \item \texttt{camareroTermina}: Cuando un camarero termina su turno.
    \end{itemize}
    
    \item La clase \texttt{HelloController} implementa esta interfaz y se encarga de:
    \begin{itemize}
        \item Gestionar las (\texttt{ObservableList}) para cada categoría de actores.
        \item Actualizar la interfaz gráfica en respuesta a los eventos recibidos.
        \item Utilizar \texttt{Platform.runLater()} para asegurar que todas las actualizaciones de UI se ejecutan en el hilo de JavaFX.
    \end{itemize}
\end{enumerate}

\subsection{Flujo de Inicio}

Al iniciar la aplicación:

\begin{enumerate}
    \item Se inicializan las listas visuales y se asocian con sus respectivas colecciones observables.
    \item Se crea la instancia de la cafetería y se le asigna el trigger (controlador).
    \item Se crean las instancias de clientes y camareros con sus parámetros específicos.
    \item Se inician los hilos de clientes y camareros con una pequeña diferencia temporal para reducir conflictos iniciales.
    \item La interfaz comienza a recibir notificaciones y a actualizarse en tiempo real, mostrando el estado de la simulación.
\end{enumerate}

\section{Casos de uso}
Ahora se describirá el diagrama de casos de uso del sistema.
\begin{center}
\begin{tabular}{ | m{10em} | m{20em}|} 
  \hline
  \textbf{Caso de uso} & \textbf{Descripción}\\
  \hline
  Botón iniciar & El usuario puede pulsar el botón en la parte superior para iniciar la simulación\\
  \hline
\end{tabular}
\end{center}

\end{document}